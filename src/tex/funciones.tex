\chapter{Funciones}

\subsection{A}

Las funciones son

es_bisiesto = function(y)
  uno_de_marzo = gauss_diasemana(1,3,y)
  veintinueve_de_febrero = gauss_diasemana(29,2,y)
  # Operación lógica de comparación (!=) aplicada a cadenas de texo
  return(uno_de_marzo != veintinueve_de_febrero)
end

Además de las comparaciones de igualdad y desigualdad (\jl{==}, \jl{!=}), también es posible utilizar las de mayor y menor (\jl{>}, \jl{<}, \jl{>=}, \jl{<=}), que se basan en el orden alfabético para determinar el resultado. La operación de multiplicación (\jl{*}) también tiene un significado con las cadenas de texto, que es la ``combinación'' de las mismas. Por ejemplo

dias = ["lunes","martes","miércoles","jueves","viernes","sábado", "domingo"]
# Días contenidos en cada mes:
# Utilizamos la función "es_bisiesto" definida anteriormente
# para ver los días de febrero
dias_mes = Dict("enero" => 31,
  "febrero" => es_bisiesto ? 29 : 28,
  "marzo" => 31,
  "abril" => 30,
  "mayo"=> 31,
  "junio" => 30,
  "julio" => 31,
  "agosto" => 31,
  "septiembre" => 30,
  "octubre" => 31,
  "noviembre" => 30,
  "diciembre" => 31)
tablahtml = "<table>\n<tr>"
# Cabecera con los nombres de los días
for dtexto in dias
  tablahtml *= "<td>$(uppercase(dtexto))</td>"
end
tablahtml *= "</tr>\n<tr>"
# Buscar la posición del primer día del mes
# y rellenar los anteriores
primerdia = gauss_diasemana(1, 2, 2012)
d = 0
for d = 1:7
  if dias[d] != primerdia
    tablahtml *= "<td></td>"
  else
    break
  end
end
# Rellenar el calendario con los números de los días
for nd = 1:29
  tablahtml *= "<td>$nd</td>"
  # Seguimos aumentando el dia de la semana d,
  # comprobando si se pasa de semana (múltiplo de 7)
  if rem(d, 7) == 0
    tablahtml *= "</tr>\n"
    if nd < 29
      tablahtml *= "<tr>"
    end
  end
  d += 1
end
# Rellenar la última semana (si está empezada) hasta el domingo
faltan = rem(d, 7) - 1
if faltan > 0
  for dfin = 1:faltan
    tablahtml *= "<td></td>"
  end
  tablahtml *= "</tr>\n"
end
tablahtml *= "</table>"
# Preguntar por el nombre del fichero para guardar el calendario
print("Escribe el nombre del fichero para el calendario")
respuesta = readline()
# Limpiar el nombre del fichero (respuesta)
# "chomp" para quitar el retorno de carro
# y "splitext" para añadir extensión ".html" si no la tiene
respuesta = chomp(respuesta)
nombre, extension = splitext(respuesta)
if isempty(extension)
  nombre *= ".html"
end
# Escribir el HTML completo
html_completo =
  "<html>
  <head>
  <meta charset=\"utf-8\">
  <title>Febrero de 2012</title>
  </head>
  <body>
  $tablahtml
  </body>
  </html>"
f = open(nombre,"w")
write(f, html_completo)
close(f)

Explicar:
*=
Multilinea
Escape: \"



La anomalía media es una posición angular desde el llamado ``pericentro''.

function amanecer_anochecer(d, m, y, long, lat, evento, criterio="normal")
	# Coseno del cénit según el criterio escogido
	cz = Dict("normal" => -0.01454,
		"civil" => -0.10453,
		"nautico" => -0.20791,
		"astronomico" => -0.30902)[criterio]
	# Formulas
	mean_anomaly = t -> 0.9856t - 3.289                             # (1)
	true_longitude = M -> M + 1.916sind(M) + .02sind(2M) + 282.634  # (2)
	right_ascension = L -> atand(0.91746tand(L))                    # (3)
	sin_decl = L -> 0.39782sind(L)                                  # (4)
	cH = (cz, sδ, cδ, sφ, cφ) -> (cz-sδ*sφ)/(cδ*cφ)                 # (5)
	local_time = (H,RA) -> H + RA - 0.065710t - 6.622               # (6)
	universal_time = (t, λ) -> t - λ                                # (7)
	# Paso 0. Calcular t:
	# tiempo aproximado del fenómeno en días desde el comienzo del año
	# El día desde el comienzo del año se calcula según la pág. B1 del
	# Almanac for Computers (1990)
	N = floor(275m/9) - floor((m+9)/12)*(1+floor((mod(y,4)+2)/3)) + d - 30
	# A N se le añaden 6 horas para el amanecer, y 18 para el anochecer,
	# menos la longitud del lugar objetivo expresada en horas
	# (λ = long (grados) *24 (horas) /360 (grados) = long/15)
	λ = long/15
	t_aprox = Dict("amanecer"=>6.0, "anochecer"=>18.0)[evento]
	t = N + (t_aprox-λ)/24
	# Paso 1. A partir del tiempo t: usar (1) y (2) para calcular
	# la anomalía media (M) y la longitud verdadera del Sol (L)
	M = mean_anomaly(t)
	L = true_longitude(M)
	# Paso 2. Calcular la ascensión recta (RA) en t a partir de (3).
	# Ajustando el resultado para que esté en el mismo cuadrante que L
	# (como atand da el resultado en ±π/2, cambiarlo si cosd(L) < 0)
	RA = right_ascension(L)
	if cosd(L) < 0
		RA += 180
	end
	# Transformar RA en horas para usarlo en (6)
	RA /= 15
	# Paso 3. Calcular el seno de la declinación solar en t (sδ) a través de (4),
	# y el coseno correspondiente (cδ, con valor positivo) para usarlos en (5)
	sδ = sin_decl(L)
	cδ = sqrt(1 - sδ^2)
	# Paso 4. Calcular el coseno del ángulo horario del Sol (cH) a partir de (5),
	# usando el coseno del cénit (cz) y el seno y coseno de la latitud (cφ, sφ).
	x = cH(cz, sδ, cδ, sind(lat), cosd(lat))
	# No considerar casos en que |x| > 1
	# (no hay amanecer o anochecer en esa latitud para ese día)
	if abs(x) > 1
		error("No hay amanecer o anochecer en el día y lugar especificados")
	end
	# Calcular el ángulo horario H como positivo o negativo según si se
	# considera el amanecer o el anochecer, y expresar en horas:
	H = Dict(
		"amanecer" => (360.0 - acosd(x))/15.0,
		"anochecer" => acosd(x)/15.0
		)[evento]
	# Paso 5. Calcular la hora local exacta del amanecer y el anochecer (T)
	# a partir de (6), convertida al rango (0-24h)
	T = mod(local_time(H, RA), 24)
	# Paso 6. Convertir a hora universal (desde el meridiano de Greenwich)
	# usando la ecuación (7)
	UT = mod(universal_time(T, λ), 24)
end


España
México
Colombia
Ciudad de México: 19.4325301, -99.1332102, UTC-6.0, 1st-Sunday April, last-Sunday October

Guadalajara: 20.676143, -103.3469982, UTC-6.0

Madrid: 40.4165, -3.70256, UTC+1.0, Last Sunday March, last Sunday October

Barcelona: 41.38879, 2.15899, UTC+1.0



Hora de amanecer:

Source:
	Almanac for Computers, 1990
	published by Nautical Almanac Office
	United States Naval Observatory
	Washington, DC 20392

Inputs:
	day, month, year:      date of sunrise/sunset
	latitude, longitude:   location for sunrise/sunset
	zenith:                Sun's zenith for sunrise/sunset
	  offical      = 90 degrees 50'
	  civil        = 96 degrees
	  nautical     = 102 degrees
	  astronomical = 108 degrees
	
	NOTE: longitude is positive for East and negative for West
        NOTE: the algorithm assumes the use of a calculator with the
        trig functions in "degree" (rather than "radian") mode. Most
        programming languages assume radian arguments, requiring back
        and forth convertions. The factor is 180/pi. So, for instance,
        the equation RA = atan(0.91764 * tan(L)) would be coded as RA
        = (180/pi)*atan(0.91764 * tan((pi/180)*L)) to give a degree
        answer with a degree input for L.


1. first calculate the day of the year

	N1 = floor(275 * month / 9)
	N2 = floor((month + 9) / 12)
	N3 = (1 + floor((year - 4 * floor(year / 4) + 2) / 3))
	N = N1 - (N2 * N3) + day - 30

2. convert the longitude to hour value and calculate an approximate time

	lngHour = longitude / 15
	
	if rising time is desired:
	  t = N + ((6 - lngHour) / 24)
	if setting time is desired:
	  t = N + ((18 - lngHour) / 24)

3. calculate the Sun's mean anomaly
	
	M = (0.9856 * t) - 3.289

4. calculate the Sun's true longitude
	
	L = M + (1.916 * sin(M)) + (0.020 * sin(2 * M)) + 282.634
	NOTE: L potentially needs to be adjusted into the range [0,360) by adding/subtracting 360

5a. calculate the Sun's right ascension
	
	RA = atan(0.91764 * tan(L))
	NOTE: RA potentially needs to be adjusted into the range [0,360) by adding/subtracting 360

5b. right ascension value needs to be in the same quadrant as L

	Lquadrant  = (floor( L/90)) * 90
	RAquadrant = (floor(RA/90)) * 90
	RA = RA + (Lquadrant - RAquadrant)

5c. right ascension value needs to be converted into hours

	RA = RA / 15

6. calculate the Sun's declination

	sinDec = 0.39782 * sin(L)
	cosDec = cos(asin(sinDec))

7a. calculate the Sun's local hour angle
	
	cosH = (cos(zenith) - (sinDec * sin(latitude))) / (cosDec * cos(latitude))
	
	if (cosH >  1) 
	  the sun never rises on this location (on the specified date)
	if (cosH < -1)
	  the sun never sets on this location (on the specified date)

7b. finish calculating H and convert into hours
	
	if if rising time is desired:
	  H = 360 - acos(cosH)
	if setting time is desired:
	  H = acos(cosH)
	
	H = H / 15

8. calculate local mean time of rising/setting
	
	T = H + RA - (0.06571 * t) - 6.622

9. adjust back to UTC
	
	UT = T - lngHour
	NOTE: UT potentially needs to be adjusted into the range [0,24) by adding/subtracting 24

10. convert UT value to local time zone of latitude/longitude
	
	localT = UT + localOffset


