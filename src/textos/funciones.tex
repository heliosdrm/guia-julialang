es_bisiesto = function(y)
  uno_de_marzo = gauss_diasemana(1,3,y)
  veintinueve_de_febrero = gauss_diasemana(29,2,y)
  # Operación lógica de comparación (!=) aplicada a cadenas de texo
  return(uno_de_marzo != veintinueve_de_febrero)
end

Además de las comparaciones de igualdad y desigualdad (\jl{==}, \jl{!=}), también es posible utilizar las de mayor y menor (\jl{>}, \jl{<}, \jl{>=}, \jl{<=}), que se basan en el orden alfabético para determinar el resultado. La operación de multiplicación (\jl{*}) también tiene un significado con las cadenas de texto, que es la ``combinación'' de las mismas. Por ejemplo

dias = ["lunes","martes","miércoles","jueves","viernes","sábado", "domingo"]
# Días contenidos en cada mes:
# Utilizamos la función "es_bisiesto" definida anteriormente
# para ver los días de febrero
dias_mes = Dict("enero" => 31,
  "febrero" => es_bisiesto ? 29 : 28,
  "marzo" => 31,
  "abril" => 30,
  "mayo"=> 31,
  "junio" => 30,
  "julio" => 31,
  "agosto" => 31,
  "septiembre" => 30,
  "octubre" => 31,
  "noviembre" => 30,
  "diciembre" => 31)
tablahtml = "<table>\n<tr>"
# Cabecera con los nombres de los días
for dtexto in dias
  tablahtml *= "<td>$(uppercase(dtexto))</td>"
end
tablahtml *= "</tr>\n<tr>"
# Buscar la posición del primer día del mes
# y rellenar los anteriores
primerdia = gauss_diasemana(1, 2, 2012)
d = 0
for d = 1:7
  if dias[d] != primerdia
    tablahtml *= "<td></td>"
  else
    break
  end
end
# Rellenar el calendario con los números de los días
for nd = 1:29
  tablahtml *= "<td>$nd</td>"
  # Seguimos aumentando el dia de la semana d,
  # comprobando si se pasa de semana (múltiplo de 7)
  if rem(d, 7) == 0
    tablahtml *= "</tr>\n"
    if nd < 29
      tablahtml *= "<tr>"
    end
  end
  d += 1
end
# Rellenar la última semana (si está empezada) hasta el domingo
faltan = rem(d, 7) - 1
if faltan > 0
  for dfin = 1:faltan
    tablahtml *= "<td></td>"
  end
  tablahtml *= "</tr>\n"
end
tablahtml *= "</table>"
# Preguntar por el nombre del fichero para guardar el calendario
print("Escribe el nombre del fichero para el calendario")
respuesta = readline()
# Limpiar el nombre del fichero (respuesta)
# "chomp" para quitar el retorno de carro
# y "splitext" para añadir extensión ".html" si no la tiene
respuesta = chomp(respuesta)
nombre, extension = splitext(respuesta)
if isempty(extension)
  nombre *= ".html"
end
# Escribir el HTML completo
html_completo =
  "<html>
  <head>
  <meta charset=\"utf-8\">
  <title>Febrero de 2012</title>
  </head>
  <body>
  $tablahtml
  </body>
  </html>"
f = open(nombre,"w")
write(f, html_completo)
close(f)

Explicar:
*=
Multilinea
Escape: \"
